% Use only LaTeX2e, calling the article.cls class and 12-point type.

\documentclass[12pt]{article}

% Users of the {thebibliography} environment or BibTeX should use the
% scicite.sty package, downloadable from *Science* at
% www.sciencemag.org/about/authors/prep/TeX_help/ .
% This package should properly format in-text
% reference calls and reference-list numbers.

\usepackage{scicite}

% Use times if you have the font installed; otherwise, comment out the
% following line.

\usepackage{times}
\usepackage{amsmath}
\usepackage{graphicx}
\graphicspath{ {./images/} }

% The preamble here sets up a lot of new/revised commands and
% environments.  It's annoying, but please do *not* try to strip these
% out into a separate .sty file (which could lead to the loss of some
% information when we convert the file to other formats).  Instead, keep
% them in the preamble of your main LaTeX source file.


% The following parameters seem to provide a reasonable page setup.

\topmargin 0.0cm
\oddsidemargin 0.2cm
\textwidth 16cm 
\textheight 21cm
\footskip 1.0cm


%The next command sets up an environment for the abstract to your paper.

\newenvironment{sciabstract}{%
\begin{quote} \bf}
{\end{quote}}


% If your reference list includes text notes as well as references,
% include the following line; otherwise, comment it out.


% The following lines set up an environment for the last note in the
% reference list, which commonly includes acknowledgments of funding,
% help, etc.  It's intended for users of BibTeX or the {thebibliography}
% environment.  Users who are hand-coding their references at the end
% using a list environment such as {enumerate} can simply add another
% item at the end, and it will be numbered automatically.

\newcounter{lastnote}
\newenvironment{scilastnote}{%
\setcounter{lastnote}{\value{enumiv}}%
\addtocounter{lastnote}{+1}%
\begin{list}%
{\arabic{lastnote}.}
{\setlength{\leftmargin}{.22in}}
{\setlength{\labelsep}{.5em}}}
{\end{list}}
\newcommand{\vect}[1]{\boldsymbol{#1}}


\title{{\it Using a convolutional neural network to classify the Street View House Number dataset}}

\author
{Andrea Ferretti\\
\normalsize{andrea.ferretti1@studenti.unimi.it}
}

% Include the date command, but leave its argument blank.

\date{}


\begin{document} 

% Double-space the manuscript.

\baselineskip24pt

% Make the title.

\maketitle 



\section*{Neural networks}
Neural networks are a family of predictors characterized by the combination of simple computational units, called neurons.
A neuron typically performs the following computation: $ g(\vect{x}) = \sigma(\vect{\omega}^T \vect{x}) $, where the elements of the vector $\vect{\omega}$ are the parameters of the neuron, $\sigma$ is a non-linear function called activation function, and $\vect{x}$ is the vector $x_0,\ldots,x_n $ where $ x_0 = 1$ and $x_i$ for $i \in \{i,\ldots,n\}$  the output computed by another neuron.
In the supervised learning setting, neurons are combined in a graph-like structure resulting in a computational network able to learn, by adjusting the parameters $\vect{\omega}$ of each neuron, the underlying mapping between data points and labels.

An example of a simple neural network is given by the on in Figure \ref{fig:mlp}

\begin{figure}[h]
    \centering
    \includegraphics[width=\textwidth]{mlp}
    \caption{A simple example of a feedforward network}
    \label{fig:mlp}
\end{figure}
%do an example with a feed forward network
%talk about how the difference for the input (no computation) and output (particular activation function depending on the task)
%explain how to optimize with backpropagation
%talk about using relu rather than sigmoid
%introduce cnn (read book on what to write)
%introduce regularization tecniques such as dropout, l2reg, batchNorm



\section*{The Street View House Number dataset}

\section*{Model training}

\section*{Results}



\end{document}




















